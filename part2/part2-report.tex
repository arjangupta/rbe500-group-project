\documentclass{article}

\usepackage{fancyhdr}
\usepackage{extramarks}
\usepackage{amsmath}
\usepackage{amsthm}
\usepackage{amsfonts}
\usepackage{tikz}
\usepackage[plain]{algorithm}
\usepackage{algpseudocode}
\usepackage{graphicx}
\usepackage{gensymb}
\usepackage{calc}
\usepackage[framed,numbered,autolinebreaks,useliterate]{mcode}
\usepackage{listings}
\usepackage{empheq}
\usepackage{enumitem}
\usepackage[font=footnotesize]{caption}
\usepackage{subcaption}

\graphicspath{{./images/}}

\usetikzlibrary{automata,positioning}

%
% Basic Document Settings
%

\topmargin=-0.45in
\evensidemargin=0in
\oddsidemargin=0in
\textwidth=6.5in
\textheight=9.0in
\headsep=0.25in

\linespread{1.1}

\pagestyle{fancy}
\lhead{\hmwkAuthorLastNames}
\chead{\hmwkClass\ \hmwkTitle}
\rhead{\firstxmark}
\lfoot{\lastxmark}
\cfoot{\thepage}

\renewcommand\headrulewidth{0.4pt}
\renewcommand\footrulewidth{0.4pt}

\setlength\parindent{0pt}

%
% Create Problem Sections
%

\newcommand{\enterProblemHeader}[1]{
    \nobreak\extramarks{}{Problem {#1} continued on next page\ldots}\nobreak{}
    \nobreak\extramarks{{#1} (continued)}{{#1} continued on next page\ldots}\nobreak{}
}

\newcommand{\exitProblemHeader}[1]{
    \nobreak\extramarks{{#1} (continued)}{{#1} continued on next page\ldots}\nobreak{}
    % \stepcounter{#1}
    \nobreak\extramarks{{#1}}{}\nobreak{}
}

\setcounter{secnumdepth}{0}
\newcounter{partCounter}

\newcommand{\problemNumber}{0.0}

\newenvironment{homeworkProblem}[1][-1]{
    \renewcommand{\problemNumber}{{#1}}
    \section{\problemNumber}
    \setcounter{partCounter}{1}
    \enterProblemHeader{\problemNumber}
}{
    \exitProblemHeader{\problemNumber}
}

%
% Homework Details
%   - Title
%   - Class
%   - Author
%

\newcommand{\hmwkTitle}{Group Assignment\ \#2}
\newcommand{\hmwkClass}{RBE 500}
\newcommand{\hmwkAuthorName}{\textbf{Joshua Gross, Arjan Gupta, Melissa Kelly}}
\newcommand{\hmwkAuthorLastNames}{\textbf{Gross, Gupta, Kelly}}

%
% Title Page
%

\title{
    \vspace{2in}
    \textmd{\textbf{\hmwkClass\ \hmwkTitle}}\\
    \vspace{3in}
}

\author{\hmwkAuthorName}
\date{}

\renewcommand{\part}[1]{\textbf{\large Part \Alph{partCounter}}\stepcounter{partCounter}\\}

%
% Various Helper Commands
%

% Useful for algorithms
\newcommand{\alg}[1]{\textsc{\bfseries \footnotesize #1}}

% For derivatives
\newcommand{\deriv}[2]{\frac{\mathrm{d}}{\mathrm{d}#2} \left(#1\right)}

% For compact derivatives
\newcommand{\derivcomp}[2]{\frac{\mathrm{d}#1}{\mathrm{d}#2}}

% For partial derivatives
\newcommand{\pderiv}[2]{\frac{\partial}{\partial #2} \left(#1\right)}

% For compact partial derivatives
\newcommand{\pderivcomp}[2]{\frac{\partial #1}{\partial #2}}

% Integral dx
\newcommand{\dx}{\mathrm{d}x}

% Alias for the Solution section header
\newcommand{\solution}{\textbf{\large Solution}}

% Probability commands: Expectation, Variance, Covariance, Bias
\newcommand{\E}{\mathrm{E}}
\newcommand{\Var}{\mathrm{Var}}
\newcommand{\Cov}{\mathrm{Cov}}
\newcommand{\Bias}{\mathrm{Bias}}

\newlength\dlf% Define a new measure, dlf
\newcommand\alignedbox[2]{
% Argument #1 = before & if there were no box (lhs)
% Argument #2 = after & if there were no box (rhs)
&  % Alignment sign of the line
{
\settowidth\dlf{$\displaystyle #1$}  
    % The width of \dlf is the width of the lhs, with a displaystyle font
\addtolength\dlf{\fboxsep+\fboxrule}  
    % Add to it the distance to the box, and the width of the line of the box
\hspace{-\dlf}  
    % Move everything dlf units to the left, so that & #1 #2 is aligned under #1 & #2
\boxed{#1 #2}
    % Put a box around lhs and rhs
}
}

\begin{document}

\maketitle

\nobreak\extramarks{Problem 1}{}\nobreak{}

\pagebreak

\begin{homeworkProblem}[Problem 1]
    \subsection{Create ROS Package for PD Controller}
    \vspace{0.1in}
    \subsubsection{Preliminary Work with Gazebo}
    Before creating the ROS package for the PD controller we performed
    some reconnaissance.\\
    \vspace{0in}\\
    When we executed \lstinline{ros2 topic list}, we saw that the
    \lstinline{/forward_effort_controller/commands} topic was part of it.
    We then executed the following command in an attempt to move the joints.
    \begin{lstlisting}
        ros2 topic pub --once /forward_effort_controller/commands std_msgs/msg/Float64MultiArray "{data: [1, 1, 1]}"
    \end{lstlisting}
    \vspace{0.15in}
    However, this took no effect. We remember from the last group assignment (Part 1)
    that we executed a very similar command to make the joints move to a specific
    position, except in that case the topic we published to was
    \lstinline{/forward_position_controller/commands}. This gave us a hint to
    the fact that Gazebo was preferring the position controller over the
    effort controller. Upon discovering the \lstinline{controller_switch.cpp} file
    in the rrbot simulation files, we realized that we must `activate' the effort
    controller in order to use it.
    Next, we did some more discovery using terminal commands. We executed
    \begin{lstlisting}
        ros2 service list -t    
    \end{lstlisting}
    This helped us see that
    \lstinline{/controller_manager/switch_controller} was an available
    service we could use. Since we supplied the -t flag to this command,
    we could also see that \lstinline{controller_manager_msgs/srv/SwitchController}
    was the type of message we had to use. To further take note of the message
    type, we executed
    \begin{lstlisting}
        ros2 interface show controller_manager_msgs/srv/SwitchController
    \end{lstlisting}
    Here we could see that \lstinline{activate_controllers} and 
    \lstinline{deactivate_controllers}
    were parameters of the message. Now we put together our findings and
    executed the following command that we constructed.
    \begin{lstlisting}
        ros2 service call /controller_manager/switch_controller controller_manager_msgs/srv/SwitchController "{activate_controllers: ["forward_effort_controller"], deactivate_controllers: [forward_position_controller, forward_velocity_controller]}"
    \end{lstlisting}

    Upon doing this, we saw the the prismatic joint drop. This means something took
    effect. We now tried our initial command,
    \begin{lstlisting}
        ros2 topic pub --once /forward_effort_controller/commands std_msgs/msg/Float64MultiArray "{data: [1, 1, 1]}"
    \end{lstlisting}
    \vspace{0.1in}
    Now we saw the robot in Gazebo move! The two revolute joints `spun' in an
    anti-clockwise direction in the XY plane, whereas the prismatic joint did
    not do anything. With some more experimentation we realized:
    \begin{itemize}
        \item Acceleration due to gravity, approximated in magnitude
        to $9.8 m/s^2$ was acting on the prismatic joint. Therefore, in order to
        make it move upward we needed to publish an effort with an acceleration
        of anything less than $-9.8 m/s^2$.
        \item In order to hold the prismatic joint
        in at any height, we needed to apply exactly $-9.8 m/s^2$ acceleration.
        The negative sign is a consequence of the direction of motion we have
        defined in our URDF file. 
        \item Since our
        joint masses are simply 1, we were able to publish a joint effort as simply
        $-9.8$, for example.
        \item We also realized that since there was no opposing force readily
        acting upon the revolute joints, if we executed an effort to them, that
        effort would keep pushing along the joints.
        \item In order to hold the revolute joints in a particular angle, we needed
        to specify that we are applying 0 effort. However this did not
        instantaneously stop the rotation of these joints, we noticed a
        de-acceleration effect before the joint came to a stop.
        \item As for implementing our package, this means that we will first
        need to execute a service call (by creating a client) to activate the
        right controller. As for executing efforts via the PD controller, we
        will need to account for gravity in case of the prismatic joint. 
    \end{itemize}

    \vspace{0.1in}
    \subsubsection{Preliminary Work for Controller}

    \begin{figure}[h]
        \includegraphics[scale=0.5]{closed-loop.png}
        \centering
        \caption{Standard PD controlled closed-loop system}
    \end{figure}

    Taking inspiration from how we have modeled PD controllers so far in
    this course, we can imagine the control for each joint as shown in
    Figure 1.\\
    \vspace{0in}\\
    Here, the goal position, $\theta_r$ will be received via a service call.
    Our package will must contain a service for this purpose.
    The current position $\theta$ is received via the \lstinline{/joint_states}
    topic. The error, $E$ is calculated as \(E = \theta_r - \theta\). This error
    is then with some untuned multiplier (gain) $K_p$ to obtain the first term of
    our controller.\\
    \vspace{0in}\\
    Next, we need to get the derivative of the error, $\dot{E}$, which could be
    expressed as
    \begin{align*}
        \dot{E} &= \dot{\theta_r} - \dot{\theta}\\
        \dot{E} &= 0 - \dot{\theta}\\
        \dot{E} &= - \dot{\theta} = -v
    \end{align*}
    Therefore the derivative of the error is the negative of $v$, the velocity of
    the joint. This velocity can be obtained from the \lstinline{/joint_states}
    topic or it can be approximated by looking at the position in the last `step'
    and the current step. Once we have $\dot{E}$, we use another untuned
    multiplier (gain) $K_d$ to obtain the second term of our 
    controller. Therefore, our controller is
    \[
        F = K_p E + K_d \dot{E} = K_p E - K_d v 
    \]
    Where F is the resultant effort that we need to apply to the joint.

    \vspace{0.2in}
    \subsubsection{Writing the ROS Package}

    Now we create the package just as we've done in our past assignments of this
    course. Inside the package, we created a \lstinline{scara_pd_controller.py}
    file.
\end{homeworkProblem}

\end{document}